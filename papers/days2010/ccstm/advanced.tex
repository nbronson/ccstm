
\subsection{Relaxed isolation}
\label{sec:unrecordedread}

Some algorithms can benefit from transactional reads that are not guaranteed to
be consistent, but
that still observe speculative stores made by the current transaction.
The inconsistent value may be used to make a heuristic decision, such as a
hash table resize, algorithm-specific knowledge may be used to guarantee
atomic behavior of the transaction despite a subsequent invalidation,
as in early release when searching a binary tree, or life cycle callbacks
may validate using specific knowledge.

Previous TM systems have provided several mechanisms for relaxing
atomicity and isolation.  Early release allows reads to be removed
from the read set prior to commit~\cite{HerlihyLMS03}.  Escape actions
suspend the current transaction temporarily~\cite{harris04exceptions}.
Open nested transactions allow the actions of a nested transaction to
be committed in a non-nested fashion.  CCSTM supports early release
and escape actions for individual accesses.  Escape action are
implemented by simply using a \code{nonTxn} bound view from inside
a transaction.  Early release is supported in a principled manner
by \type{Source.Bound}\code{.releasableRead}.  This method returns a
\type{ReleasableRead} instance that bundles the requested value with
a method that removes the record of the access from the transaction's
read set.  This interface eliminates the danger that an algorithm will
remove a read that it did not perform, but it still requires careful
reasoning to guarantee correctness after the read has been released.

As an alternative to a releasable read, CCSTM includes a new
abstraction, \code{unrecordedRead}.  This method performs a
transactional read, but instead of adding an entry to the read set it
bundles the read's meta-data into an \type{UnrecordedRead} instance.
The caller may then use this instance to manually validate that the
returned value is still valid.

Like many STMs, CCSTM performs transactional reads by associating a
version number with each managed memory location, recording the version
prior to a transactional read, and checking during validation that the
version number remains unchanged.  An \type{UnrecordedRead} contains the
read value and the prior version, but rather than automatically validating
the read during commit, validation is exposed to the programmer via the
method \code{stillValid}.  An unrecorded read is considered to still be
valid if the only changes that have been made to the referenced memory
location were performed by the read's transaction.  This definition also
produces a meaning for unrecorded reads of the \code{nonTxn} bound view:
\code{stillValid} will return true only if no change has been made to the
managed value.  This leverages the STM's meta-data to solve
the ABA problem\footnote{The ABA problem is when an observer falsely
concludes that a value has not changed, because the watched value went
from A to B, then back to A.}.

\subsection{Semantic conflict detection for reads}
\label{sec:map}

Unrecorded reads can be paired with life cycle callbacks to implement
Abstract Nested Transactions~\cite{harris07abstract}.  For the simple case where
a single transactional read is modified by an idempotent function, \type{Ref}
provides
\code{map[}\typeparam{Z}\code{](f: }\typeparam{T}\code{ => }\typeparam{Z}\code{): }\typeparam{Z}.
A transactional call to
\code{x.map(f)} returns the same value as \code{f(x.get)}, but no rollback
is triggered by a conflicting write to \code{x} if the result of the mapping
does not change.  Without this
semantic conflict detection, the STM must initiate rollback any time \code{x}
is changed concurrently, even if that change is masked by the application of
\code{f}.

Consider the branch taken by Figure~\ref{fig:example:ccstm}'s
Line~\ref{fig:example:ccstm:D} during an attempt to withdraw 1,000
\type{Money} from an account with a balance of 500.  At this point
\code{\_balance} will be included in the transaction's read set, so a
concurrent deposit of 100 will cause the withdrawal transaction to be
rolled back despite not changing the withdrawal's outcome.  Although only
a single bit of information about the balance was retained, the STM
must conservatively assume that any change invalidates the speculative
execution.  If we move the inequality application into a predicate
applied by \code{map}, however, the STM can recompute that bit and
determine that the withdrawal transaction is still valid:
\lstset{numbers=none}
\begin{lstlisting}
// if (_balance() < m) ...
if (_balance.map(_ < m)) ...
\end{lstlisting}
\lstset{numbers=left}
By allowing the programmer to express more of her intention to CCSTM,
\code{map} can avoid rollbacks and lead to better scalability, especially for
transactions that have already performed a substantial quantity of work.
Figure~\ref{fig:map} shows how \code{map} may be implemented using
\code{unrecordedRead} and a validation handler.

\begin{figure}
\begin{lstlisting}{name=Code}
class Ref[T] {
  def map[Z](f: T => Z)(implicit t: Txn): Z = {
    val u0 = unrecordedRead
    val result = f(u0.value)
    addReadResource(new Txn.ReadResource {
      var u = u0 // latest unrecorded read
  
      def valid(t: Txn) = {
        if (u.stillValid) {
          true
        } else {
          // reread and compare to original
          u = unrecordedRead
          (result == f(u.value))
        }
      }
    }, 0, false)
    result
  }
}
\end{lstlisting}

\caption{An implementation of \type{Ref}\code{.map} that combines the
\code{unrecordedRead} primitive with a read-resource life cycle callback.
Conflicting changes to the reference do not require the transaction to
be rolled back if \code{f(get)} does not change.}

\label{fig:map}
\end{figure}


\subsection{All of the ways to read or write a CCSTM reference}

What follows is the complete list of the access operations provided by
\type{Ref.Bound}.  Many of these methods have equivalents in \type{Ref}
that take an implicit \type{Txn}, although to reduce the API's surface
area some methods are not mirrored.  All of the methods are defined
for both transactional and non-transactional contexts, even if they are
useful for only one of those.

{
\setlength{\leftskip}{12pt}
\setlength{\parindent}{-12pt}
\setlength{\parskip}{3pt}

\vspace{2pt}
\type{\bfseries Source.Bound:}

\code{apply(): }\typeparam{T }\\ Equivalent to \code{get}.

\code{get: }\typeparam{T }\\ Reads the value managed by the
bound \type{Ref}.  If this view is bound to a non-transactional context,
the read will be strongly atomic and isolated with respect to all
transactions, and will linearize before returning.

\code{map[}\typeparam{Z}\code{](f: }\typeparam{T}\code{ => }\typeparam{Z}\code{): }\typeparam{Z }\\
Returns \code{f(get)}, possibly reevaluating \code{f}
to avoid rollbacks (\code{f} must be idempotent).

\code{await(p: }\typeparam{T}\code{ => }\type{Boolean}\code{) }\\ Blocks
until \code{p(get)} is true.  Transactional contexts block by rolling
the transaction back using \code{retry}, the modular blocking primitive.
Non-transactional contexts just block.

\code{unrecordedRead: }\type{UnrecordedRead}\code{[}\typeparam{T}\code{] }\\
Returns an instance that wraps the value that would be returned by
\code{get}, but does not add anything to the transaction's read set
(if any).

\code{releasableRead: }\type{ReleasableRead}\code{[}\typeparam{T}\code{] }\\
Reads the value managed by the bound \type{Ref}, and returns that
value in an instance that allows the corresponding read set entry (if any)
to be removed.

\vspace{3pt}
\type{\bfseries Sink.Bound:}

\code{:=(v: }\typeparam{T}\code{) }\\ Equivalent to \code{set(v)}.

\code{set(v: }\typeparam{T}\code{) }\\ Updates the value managed by the
bound \type{Ref}.  If this view is bound to a non-transactional context,
this method will linearize the store before returning.

\code{tryWrite(v: }\typeparam{T}\code{): }\type{Boolean }\\ Immediately
performs an update and returns true, or does nothing and returns false.

\vspace{3pt}
\type{\bfseries Ref.Bound extends Source.Bound with Sink.Bound:}

\code{readForWrite: }\typeparam{T }\\ Returns the same value as
that returned by \code{get}, but adds the bound \type{Ref} to the write
set of the transaction context, if any.

\code{getAndSet(v: }\typeparam{T}\code{): }\typeparam{T }\\ Atomically
invokes \code{set(v)} and returns the old value.

\code{compareAndSet(b: }\typeparam{T}\code{, v: }\typeparam{T}\code{): }\type{Boolean }\\
Atomically performs
\code{(b == get) \&\& \{ set(v); }\keyword{true}\code{ \}}

\code{compareAndSetIdentity(b: }\typeparam{T}\code{, v: }\typeparam{T}\code{): }\type{Boolean }\\
Atomically performs
\code{(b }\keyword{eq}\code{ get) \&\& \{ set(v); }\keyword{true}\code{ \}}

\code{weakCompareAndSet(b: }\typeparam{T}\code{, v: }\typeparam{T}\code{): }\type{Boolean }\\
Either performs \code{compareAndSet} or returns false.

\code{weakCompareAndSetIdentity(b: }\typeparam{T}\code{, v: }\typeparam{T}\code{): }\type{Boolean }\\
Either performs \code{compareAndSetIdentity} or returns false.

\code{transform(f: }\typeparam{T}\code{ => }\typeparam{T}\code{) }\\
Atomically replaces the stored value \code{v} with \code{f(v)}.

\code{getAndTransform(f: }\typeparam{T}\code{ => }\typeparam{T}\code{): }\typeparam{T }\\
Atomically replaces the value \code{v} stored in the \type{Ref}
with \code{f(v)}, returning the old value.

\code{tryTransform(f: }\typeparam{T}\code{ => }\typeparam{T}\code{): }\type{Boolean } \\
Immediately atomically transforms this reference and returns true,
or returns false.

\code{transformIfDefined(pf: }\type{PartialFunction}\code{[}\typeparam{T}\code{,}\typeparam{T}\code{]):}
\type{Boolean }\\
Atomically replaces the value \code{v} stored in the bound \type{Ref}
with \code{f(v)} if \code{pf.isDefinedAt(v)}, returning true, otherwise
leaves the value unchanged and returns false.

}

\subsection{Removing storage indirection in user classes}

\todo{TxnFieldUpdater}

\subsection{Conditional retry}

CCSTM supports the \code{retry} and \code{orElse} primitives introduced by
Harris et al. in Haskell's STM~\cite{harris05ctm}, although the current
lack of partial rollback when nesting makes them less expressive than the original.
The \code{retry} primitive causes the surrounding transaction to be rolled
back, but retry is postponed until at least one of the values read by
the transaction has changed.  \code{orElse} combines two transactions,
attempting the second if the first calls \code{retry}, then blocking
both transactions if the second calls \code{retry}.  Intuitively, a call
to \code{retry} is a dead end; the STM will restart the transaction
only after it might take a different path.  Similarly, \code{orElse}
composes two alternatives that are each satisfactory, and requests that
whichever one can avoid the dead end should be executed.

Currently, CCSTM encodes \code{retry} as a method of the \code{STM} object, and 
combines composition and atomic execution of a sequence of atomic blocks into
\code{STM.atomicOrElse[}\typeparam{Z}\code{](blocks: (}\type{Txn}\code{ => }\typeparam{Z}\code{)*): }\typeparam{Z}.
While we have experimented with an implicit conversion from 
\type{Txn}\code{ => }\typeparam{Z} to an \type{AtomicBlock} that provides a
rich interface, we have not yet found a syntax that works well.  If
\code{retry} is used without \code{orElse}, then the normal \code{STM.atomic}
method may be used.

As a (hopefully) contrived example,
the bank could use modular blocking to withdraw money from exactly one of a number of
accounts, blocking until success:
\lstset{numbers=none}
\lstset{xleftmargin=0.125in}
\begin{lstlisting}
class Account {
  ...
  def withdrawOrRetry(m: Money
        )(implicit t: Txn) {
    if (_balance() < m) STM.retry
    _balance := _balance() - m
  }
}
object Account {
  def fromAny(m: Money, srcs: Account*) {
    val blocks = srcs map { s =>
        { (t: Txn) => s.withdrawOrRetry(m)(t) } }
    STM.atomicOrElse(blocks: _*)
  }
}
\end{lstlisting}
\lstset{numbers=left}
\lstset{xleftmargin=0.25in}

%Ref eq vs ==

