
An experimental feature such as software transactional memory should
strive to impose only negligible costs on code that does not use it.
Runtime performance costs are the most obvious, but extra complexity in
the compiler, libraries, and language rules should be minimized.
A pay-as-you-go philosophy facilitates incremental adoption, it allows
multiple implementations to coexist, and it reduces the penalty for
failure.

One popular and reasonable interface design for transactional memory
is to mimic lock-based critical regions.  Users of such an STM declare
the beginning and the end of an atomic block, and all memory accesses
that occur within the dynamic scope of the block are transparently
redirected to the STM.  For a VM language like Scala this redirection
can be introduced by the VM's JIT, by bytecode rewriting at class
load time, or during the initial
compilation of the high-level language.  The dynamic scoping of such an approach,
however, means that it is generally not possible to
limit instrumentation to only classes that are used in an atomic block.
An STM that is deeply integrated into the VM's JIT can minimize the
performance and code bloat impacts of the instrumentation by performing it
lazily~\cite{??}, but the required engineering effort to add this support
to a production quality VM is prohibitively large.  Instrumentation of
the bytecode at compilation or class loading has the lowest engineering
cost, but results in two copies of each method.
This is the strategy adopted by the Multiverse~\cite{multiverse} and
Deuce STM~\cite{deucestm} STMs for the Java language.
While this cost may eventually be considered acceptable, it places a
high hurdle to integration into Scala's standard library.
An additional drawback of
an instrumentation approach is that it is not composable.  If module
\textit{A} is constructed with the STM \textit{S} and module \textit{B}
is constructed with the STM \textit{T}, then \textit{A} and \textit{B}
can't be used in the same program.

The alternative approach adopted by CCSTM is to require the programmer to perform
explicit calls to the STM.  While less convenient for simple uses, this
limits performance side-effects on code that does not use atomic blocks, and it
allows the STM to be constructed entirely as an unprivileged library.
When coupled with an STM design that does not assume it is managing all
threads, the result is a pay-as-you-go transactional memory suitable
for experimentation and incremental adoption.

Scala's flexible syntax makes a library-only STM tractable.  Operator
overloading makes transactional loads and stores more concise,
and implicit parameters allow the current transaction context to be
threaded through the code without explicitly including it in each call.
The resulting STM can be considered to be an embedded domain-specific
language (DSL) for expressing optimistic concurrency.



\subsection{Strong isolation}

One of the benefits of the reference-based approach is that it avoids
isolation problems between transactional and non-transactional accesses to
the same memory location, without requiring any changes to the underlying
type system.

At its most basic, a software transactional memory is a way of isolating
a group of memory accesses and verifying that those accesses are
equivalent to some serial execution.  The STM manages reads and writes
by redirecting them to \textit{barriers}, code fragments that actually
implement the atomicity and isolation.  These properties can only be
guaranteed, however, if no non-transactional code bypasses the barriers
and accesses a memory location directly.

There are three potential responses to the weak isolation between direct memory
accesses and concurrent transactions:
\begin{packed_itemize}

\item The system can provide strong isolation and atomicity by redirecting
all memory accesses to barriers, even non-transactional accesses.
While there has been some research in using dynamic recompilation
to reduce the performance penalty of strong isolation, these require
either deep integration with the VM's JIT~\cite{schneider08dynamic}
or a substantial warmup period~\cite{bronson09dbo}.

\item The system can declare that a conflicting concurrent access
from both inside and outside a transaction is an error.  This doesn't
sound too onerous, but the optimistic nature of transactions means that
failed speculations must also be considered: inconsistent transactions
may execute conflicting accesses from an impossible branch, or they
may execute conflicting accesses after they have become doomed.
Restrictions on commit order can prevent some of the most surprising
behaviors~\cite{sgla08}, but the resulting systems still require
whole-program reasoning to guarantee correctness.  The privatization
problem and its dual, the publication problem, refer to isolation failure
for specific useful idioms.

\item The system can use types to disallow direct access to any memory
location that might be touched transactionally~\cite{moore08semantics}.
This can take the form of extending the types and access rules on
% todo: add a citation here
normal mutable memory locations, or
of boxing all transactionally-managed data inside some sort of cell,
as in Haskell~\cite{harris05composable} or Clojure~\cite{hickey08clojure}.  We refer to the latter
approach as a reference-based STM.

\end{packed_itemize}

Scala favors safety and compile-time checking of program correctness,
so the authors are of the opinion that it is only natural to employ types
to avoid the problems of weak isolation.  In the long term an extension
to Scala's types seems possible, but in the short term a library-based
approach seems the most practical.

\subsection{Irrevocable actions and structural conflicts}

One of the side effects of an alternate syntax for transactional barriers
is that it avoids creating the impression that the STM can magically
parallelize all existing sequential code, or that atomic blocks are
always a better replacement for locks.  There are both semantic and
practical reasons why this is not the case.  The semantic problems come
from actions that the STM cannot isolate or undo, such as I/O or calls to
external libraries.  In the absence of any additional information about
potential conflicts, the only way to execute safely is to serialize.
The practical problem with executing code that was not designed to
be executed inside an atomic block is that such code often contains
incidental shared accesses that the STM must treat as conflicts.
An example of this is the size field of a collection, which is often
accessed by every mutating operation.  Unless care is taken to distribute
the size over multiple memory locations, no concurrency will actually
be available.

\subsection{Opacity}

A subtle issue with STM is that, unless special care is taken, only
committed transactions are guaranteed to be consistent.  Speculative
transactions may observe an inconsistent state and only subsequently
detect that they should roll back.  These `zombies' can produce
surprising behavior by taking impossible branches or performing
transactional accesses to the wrong object.  This problem is greatly
magnified in a reference based STM, because the STM cannot provide a
sandbox that isolates all actions taken by the zombie.  The read of a
single impossible value may produce an infinite loop, so a transparent
STM must either prevent inconsistent reads or instrument back edges
to periodically revalidate the transaction.  Only the first option is
available to an STM implemented as a library.

The TL2~\cite{dice06tl2} and LSA~\cite{riegel06lsa} algorithms
demonstrated how to use a global time-stamp to efficiently validate
a transaction after each read, guaranteeing consistency for all
intermediate states.  This correctness property was later formalized
as \textit{opacity}~\cite{guerraoui08opacity}.  CCSTM uses the
SwissTM~\cite{dragojevic09swisstm} algorithm, which guarantees opacity.

%STMs often attempt to mimic the syntax used for lock-based critical
%regions.  The beginning and end of the atomic block are declared, and all
%memory accesses that occur within the dynamic scope of the atomic block
%are transparently redirected to read or write barriers.  This has the
%advantage that transactional accesses inherit the programming language's
%syntax for reads and writes, which is both concise and familiar.
%This familiarity is both a blessing and a curse, however.  It creates an
%expectation that the STM can safely execute arbitrary existing sequential
%code, which creates both semantic and performance challenges, and it
%makes it more difficult for the programmer to convey semantic information
%to the STM, which reduces opportunities for compile-time checking and
%run-time scalability and performance improvements.

%A Scala STM implemented inside the VM's JIT is likely to have the best
%performance.  Past research has identified profitable transformations
%that can be made by the optimizer if it has knowledge of barrier
%usage~\cite{?}.  This approach is not a pragmatic one, however, because
%the engineering effort required to produce a production quality VM with
%this capacity is feasible only for a few organizations.

%Bytecode rewriting is a promising technique.  Several bytecode rewriting STMs
%have been described for the JVM, including Multiverse~\cite{multiversewebsite},
%Deuce STM~\cite{??}, and AJ~\cite{popl09bronson}.  For Scala, this approach
%has the advantage of allowing transactional execution of Java code,
%including the Java standard libraries.

%Bytecode rewriting has several drawbacks, however.  Because transactional
%execution is dynamically scoped, it is not known at class loading time
%whether or not transactional versions of an object or a method will
%be required.  This means that all methods must be duplicated, adding
%class loading overhead and increasing the size (and therefore decreasing
%the spatial locality) of the VM's class data structures.  The bytecode
%rewriter must be given special privileges, which complicates deployment
%and limits the environments in which it may be used.

