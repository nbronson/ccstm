
CCSTM's implementation is modeled on SwissTM~\cite{swisstm}.  Version
management is lazy, but write permission is acquired eagerly.  Time-stamps
are allocated 51 bits, making CCSTM immune from counter overflow in all
but the most demanding production environments.

\subsection{Meta-data indirection}

Meta-data for a managed memory location consists of a single \keyword{long}.
It is assumed that each memory location maps to a unique meta-data value, but
not vice versa.  This allows objects with multiple fields to use
a single piece of meta-data, and it allows arrays to choose a variety of
granularities of conflict detection.  While some optimizations are possible for
situations where the data-to-meta-data mapping is one-to-one, in informal
experiments we found that the benefits were smaller than the additional
indirection costs.

\type{Ref}s perform their accesses to both data and meta-data through
methods of an internal trait called a \type{Holder}.  This indirection
allows multiple storage strategies to be easily provided, which can
yield an important reduction in the number of live objects in the VM.
For example, if the static or manifest type of the initial value is known
to be an \keyword{int}, then the \type{Ref} factory method will return
a reference whose holder stores the value in an unboxed form.  As a
more extreme example, CCSTM provides a transactional array-like class
that internally uses one array for values and one array for meta-data,
eliminating the $n$ intermediate objects that would be required by an
\ytype{Array}{\xtype{Ref}{A}}.  In some cases these storage optimizations
return CCSTM to par with a VM-integrated STM, in some cases they make
CCSTM's representation more compact.

\subsection{Global time-stamp optimizations}

To reduce contention on the shared time-stamp, CCSTM uses TL2's GV6
scheme~\cite{dice06tl2}.  This mechanism is based on the observation
that, while committed values must be given a time-stamp later than the
version clock that was present at the beginning of the commit, it is
not required that the global clock is actually advanced.  Advancing the
global clock reduces the need for validation in later transactions,
but when many threads are using the STM, this goal is satisfied even if
only a fraction of transactions attempt to advance the current time.

CCSTM performs a novel additional optimization to reduce the overhead
of non-transactional accesses.  Unlike a transaction, a solitary
strongly-isolated read or write in a TL2-style STM does not need to
sample the global clock to provide opacity.  This means that we can allow
a sequence of non-transactional writes to advance a reference's time-stamp to
an arbitrary point in the future, without advancing the global time-stamp.
If a transaction attempts to read such a far-future value it handles
it via the normal GV6 mechanism, by advancing the global time-stamp and
then revalidating.  To limit the potential impact of these booby-trapped
references, we only allow non-transactional writes to advance time-stamps
a limited distance into the future.  Even a small window (CCSTM defaults
to 8) dramatically reduces contention on the global time-stamp.

\subsection{Avoiding starvation}

Optimistic concurrency control is vulnerable to the \textit{starving
elder} problem, in which a large transaction can never be committed because
it is continually violated by small transactions.  CCSTM uses a simple
contention manage scheme to prevent this.  Each execution attempt is
assigned a random priority that is used to resolve write-write conflicts.
If a transaction has not yet begun to commit, then a higher priority
transaction may doom it and steal its locks.  In addition, transactions
that have already failed several times enter a `barging' mode in which
they acquire write permission during reads.  The result is that even
large transactions will eventually succeed, because they will eventually
receive the highest priority in the system.

\subsection{Polite blocking}

An important design goal for CCSTM is support for incremental use inside
a larger application.  This means that exponential back-off is not a
suitable mechanism for blocking.  Many STMs target parallel speedups
for CPU-bound applications.  Assumed (often implicitly) is that all
threads belong to the STM and that the number of software threads can be
chosen to match the number of hardware execution contexts.  CCSTM makes
neither of these assumptions, and so takes care to block using the normal
synchronization primitives of the underlying VM.

Blocking may be required to obtain write permission, or because of an
explicit use of the \code{retry} primitive.  Writers and waiters must
agree on a condition variable that will be used to signal that the
waiter should re-attempt whatever action led to their choice to block.
If the set of condition variables is too small, there will be many
spurious wakeups.  If the set is too large, then transaction commit may need to
perform a large amount of extra work.

Accesses that are blocked by another transaction await notification
on the \type{Txn} instance itself.  No such instance is available for
threads blocked by a non-transactional write, or that are performing a
conditional retry, so the system also maintains 64 lists we refer to as
`wakeup channels'.  These channels contain a list of pending wakeups,
which are single-shot gates (compare to Java's \type{CountDownLatch}
with a count of 1).  Each memory location is associated with a wakeup
channel by hashing its identity.  To await the modification of a
memory location, a thread enqueues a new pending wakeup instance,
sets a `wakeup pending' bit in the location's meta-data, rechecks the
blocking condition, and then puts itself to sleep on the gate.  If an
update notices the wakeup pending bit, it triggers and removes all of
the pending wakeups for the corresponding channel.  A thread may wait on
multiple memory locations simultaneously by enqueuing its pending wakeup
instance to multiple channels.  The choice of 64 wakeup channels makes
it easy to accumulate the effects of a transaction in a \keyword{long}.
If a system makes extremely heavy use of the \code{retry} mechanism by
having many blocked threads, a larger number of channels may be required.

\subsection{JVM versus CLR}

Scala is designed to target both the JVM and the CLR virtual
machines.  In its current implementation, CCSTM uses the following
classes from \type{java.util.concurrent} to implement atomic
compare-and-swap for fields and array elements: \type{AtomicInteger},
\type{AtomicLong}, \type{AtomicLongArray}, \type{AtomicReferenceArray},
\type{At\-om\-ic\-LongFieldUpdater}, and \type{AtomicReferenceField\-Up\-dat\-er}.
The authors are not experts in the CLR memory model, but we believe
that it would be straightforward to retarget CCSTM to the CLR by using
methods in \type{System::Threading::Interlocked}.

