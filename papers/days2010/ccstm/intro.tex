The proliferation of multi-core processors means that more programmers are
being thrust into the difficult world of shared memory multi-threading.
\todo{a couple sentences more of boilerplate motivation}

Software transactional memory (STM) provides a compelling alternative to
locks for managing access to shared mutable state.  STM's declarative
atomic blocks are free from deadlock, are composable, and do not
require elaborate fine-grained decomposition to yield scalability.
These benefits stem from an optimistic execution strategy that includes
rollback and retry, allowing the runtime to attempt concurrent execution
based only on a likelihood of success.  STMs bridge the gap between their
simple programming model and their speculative execution by isolating
uncommitted transactions from the rest of the system.  Most of the
difficulty in integrating an STM into a language comes from tradeoffs
between the overhead, fidelity, and complexity of this isolation.

In this paper we describe the design of CCSTM, a library-based STM for
Scala.  CCSTM deliberately sidesteps many of the performance and semantic
difficulties common in software implementations of transactional memory
by limiting its focus: CCSTM is a tool for use by parallel programmers
that wish to build algorithms and data structures using optimistic
concurrency control, not an all-encompassing concurrent programming
model or a mechanism for automatic parallelization of arbitrary code.

The most fundamental design choice for CCSTM was the decision to
implement it entirely as a Scala library.  Unlike STMs that instrument
normal loads and stores, transactional accesses in CCSTM are explicit
method calls to a Scala \xkeyword{trait}\footnote{Property accessors
may used to eliminate the syntactic overhead for many situations.}.
We refer to the resulting STM as `reference-based', because all memory
locations managed by the STM are boxed inside transactional references.
Both transactional and non-transactional access to the managed references
goes through methods implemented by instances of this \xtypeA{Ref}.

While a reference-based STM imposes a syntactic burden for simple loads and
stores, it also makes the STM more expressive.  A \xtype{Ref} provides
a first-class object that names a memory location, something that would
otherwise require reflection in Scala.  In addition, the reference provides a
convenient namespace for additional STM functionality, such as semantic
conflict detection or transformation by an associative function.
%As a
%practical benefit, the extra level of indirection provided by \xtype{Ref}
%allows multiple implementations, allowing tradeoffs between the cost of
%validation and the likelihood of rollback.

CCSTM's second departure from the typical STM interface is to bind the
transaction scope statically, rather than dynamically.  Transactional access
methods in \xtype{Ref} take an implicit parameter of type \xtype{Txn}, and
perform their accesses in the context of that transaction.  This decision
was made for pragmatic reasons, because performing a dynamically-scoped
lookup on each transactional access would be prohibitively expensive.

A useful parallel can be made between Haskell's STM [[ref]] and CCSTM.
Haskell's \xtype{TVar} corresponds to instances of type \xtype{Ref}.
\xtype{Txn} is not a monad, but it proliferates through STM-enabled
methods in exactly the same way as the \xtype{STM} monad.

In this paper, we:
\begin{packed_enum}

\item Describe CCSTM, a reference-based STM for Scala.  CCSTM focuses on
helping parallel programmers build optimistically concurrent algorithms
and data structures, while restricting itself to implementation techniques
that do not interfere with components of the system that do not use it
(Section~\ref{fig:?}).

\item Introduce `unrecorded reads', a primitive that can be coupled with
transaction lifecycle callbacks to provide semantic conflict detection.  We use
unrecorded reads to add a \xcode{map(f)} method to \xtype{Ref} that performs
conflict detection on the result of applying \xcode{f}, rather than on the
transactional value (Section~\ref{fig:?}).

\item Summarize the discussions that led from the original design goal to
the current syntax.  We point out the parts that work well and the parts that
are burdensome, and hypothesize about ways to address the latter
(Section~\ref{fig:?}).

\item Compare the performance of CCSTM to Multiverse and DeuceSTM, STMs for
the JVM that use bytecode rewriting.  We find that CCSTM's implementation as an
unpriveleged library does not impose a performance penalty
(Section~\ref{fig:?}).  \todo{is this true}

\end{packed_enum}

