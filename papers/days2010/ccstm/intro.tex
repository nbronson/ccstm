The proliferation of multi-core processors means that more programmers are
being thrust into the difficult world of shared memory multi-threading.
Software transactional memory (STM) provides a compelling alternative to
locks for managing access to shared mutable state; STM's declarative
atomic blocks are free from deadlock, are composable, and do not
require elaborate fine-grained decomposition to yield scalability.

%The benefits of memory transactions stem from an optimistic execution
%strategy that includes rollback and retry.  The ability to recover from
%error allows the runtime to
%attempt concurrent execution without a guarantee that it is possible.
%STMs bridge the gap between their simple programming model and their
%speculative execution by isolating uncommitted transactions from the
%rest of the system.  Most of the difficulty in integrating an STM into
%a language comes from tradeoffs between the overhead, fidelity, and
%complexity of this isolation.

In this paper we describe the design of CCSTM, a library-based STM
for Scala.  CCSTM deliberately sidesteps many of the 
semantic difficulties common in software implementations of transactional
memory, by limiting its focus.  We view CCSTM as a domain-specific language
(DSL)
for use by parallel programmers that wish to build algorithms and data
structures using optimistic concurrency control.  CCSTM is not a drop-in replacement for
locks, an all-encompassing concurrent programming model, or a mechanism
for automatic parallelization of arbitrary code.

The most fundamental design choice for CCSTM was the decision to implement
it entirely as a Scala library.  Unlike STMs that transparently instrument
all loads and stores of shared mutable state, transactional accesses in
CCSTM are explicit method calls to a Scala \keyword{trait}\footnote{Custom
accessor methods may used to eliminate the syntactic overhead for some
situations.}.  We refer to the resulting STM as `reference-based',
because all memory locations managed by the STM are boxed inside
transactional references.  Both transactional and non-transactional
access to the managed references goes through methods implemented by
instances of this \xtype{Ref}{A}.

While a reference-based STM imposes a syntactic burden for simple loads
and stores, it provides advantages.  Encapsulating transactionally-managed
data eliminates the weak isolation issues that plague transparent STMs.
In addition, the \type{Ref} provides a first-class object that names a
memory location, which enables CCSTM to provide additional functionality
to the user in a natural way.

%As a
%practical benefit, the extra level of indirection provided by \type{Ref}
%allows multiple implementations, allowing tradeoffs between the cost of
%validation and the likelihood of rollback.

CCSTM's second departure from the typical STM interface is that it uses
a hybrid of static and dynamic transaction scoping.  Static resolution
is used for methods on \type{Ref}, while dynamic lookup is used to
perform nesting when entering a new atomic region.  Static resolution
is accomplished by an implicit \type{Txn} parameter passed to methods
of \type{Ref}.  This results in very low overheads, which is important
because barrier methods are called very frequently.  Dynamic resolution
uses a \type{ThreadLocal} to determine if a new atomic region should
be nested, which reduces the need to thread the implicit \type{Txn}
through all of the user's methods.

%This decision
%was made for pragmatic reasons, because performing a dynamically-scoped
%lookup on each transactional access would be prohibitively expensive.
%A useful parallel can be made between Haskell's STM~\cite{harris05composable} and CCSTM.
%Haskell's \type{TVar} corresponds to instances of type \type{Ref}.
%\type{Txn} is not a monad, but it proliferates through STM-enabled
%methods in exactly the same way as the \type{STM} monad.

In this paper:
\begin{packed_enum}

\item We describe CCSTM, a reference-based STM for Scala.  CCSTM focuses
on helping parallel programmers build optimistically concurrent algorithms
and data structures, while restricting itself to implementation techniques
that do not interfere with components of the system that do not use it
(Section~\ref{sec:ref}).

\item We introduce \code{unrecordedRead}, a new STM primitive that relaxes
read atomicity while allowing manual validation.  To demonstrate its
use, we implement \type{Ref}\code{.map}, a simple but restricted form of
Abstract Nested Transaction~\cite{harris07abstract}.  \code{x.map(f)}
returns the same value as \code{f(x.get)}, but does not require the
transaction to roll back if \code{x} is changed concurrently but
\code{f(x.get)} remains the same (Sections~\ref{sec:unrecordedread}
and~\ref{sec:map}).

\item We briefly overview CCSTM's implementation, including a
novel optimization to reduce the global time-stamp contention of
non-transactional isolated writes (Section~\ref{sec:impl}).

\item We summarize some of the discussions that led from the original
design goal to the current syntax.  We point out the parts that work
well and the parts that are cumbersome, and hypothesize about ways to
address the latter (Section~\ref{sec:discussion}).

\item We compare the performance of CCSTM to Deuce STM and Multiverse,
STMs for the JVM that use bytecode rewriting~\cite{deucestm,multiverse}.
We find that CCSTM's implementation as an unprivileged library
does not impose a significant performance or scalability penalty
(Section~\ref{sec:perf}).

\end{packed_enum}

