%\documentclass{sigplanconf}
\documentclass[preprint]{sigplanconf}

\usepackage{amsmath}
\usepackage{amssymb}
\usepackage{textcomp}
\usepackage[usenames]{color}
\usepackage{graphicx}
\usepackage{url}
\usepackage{rotating}
\usepackage{multirow}
\usepackage{booktabs}
\usepackage{listings}
\usepackage{array}

\newenvironment{packed_enum}{
\begin{enumerate}
  \setlength{\topsep}{0pt}
  \setlength{\itemsep}{2pt}
  \setlength{\parskip}{0pt}
  \setlength{\parsep}{0pt}
}{\end{enumerate}}

\newenvironment{packed_itemize}{
\begin{itemize}
  \setlength{\topsep}{0pt}
  \setlength{\itemsep}{2pt}
  \setlength{\parskip}{0pt}
  \setlength{\parsep}{0pt}
}{\end{itemize}}

\newcommand{\todo}[1]{{\color{red} \bf [TODO: #1 ]}}


\lstset{
  language=Java,
  basicstyle=\small\tt,
  keywordstyle=\small\tt,
  numbers=left, 
  numberstyle=\small\tt,
  commentstyle=\small\tt,
  columns=fullflexible,
  xleftmargin=0.25in,
  firstnumber=auto,
  escapechar=\#
}



\begin{document}

\conferenceinfo{PPoPP'10,}{January 9--14, 2010, Bangalore, India.}
\CopyrightYear{2010}
\copyrightdata{978-1-60558-708-0/10/01}

\providecommand{\e}[1]{\ensuremath{\times 10^{#1}}}

\newcommand{\xtitle}[0]{A Practical Concurrent Binary Search Tree}
\newcommand{\xterms}[0]{{Algorithms, Performance}}
\newcommand{\xkeywords}[0]{{Optimistic Concurrency, Snapshot Isolation}}
%\titlebanner{[WORK IN PROGRESS: Do Not Circulate!]}        % These are ignored unless
\preprintfooter{\xtitle}  % 'preprint' option specified.

\title{\xtitle}

\authorinfo{Nathan G. Bronson\and Jared Casper \and Hassan Chafi\and Kunle Olukotun}
           {Computer Systems Laboratory\\
            Stanford University}
           {\textit{\{nbronson, jaredc, hchafi, kunle\}@stanford.edu}}

\pdfinfo{
  /Title    (\xtitle)
  /Author   (Nathan G. Bronson, Jared Casper, Hassan Chafi and Kunle Olukotun)
  /Subject  (\xterms)
  /Keywords (\xkeywords)
}

\maketitle

\begin{abstract}

We propose a concurrent relaxed balance AVL tree algorithm that is
fast, scales well, and tolerates contention.  It is based on optimistic
techniques adapted from software transactional memory,
but takes advantage of specific knowledge of the the algorithm to reduce
overheads and avoid unnecessary retries.  We extend our algorithm with
a fast linearizable \texttt{clone} operation, which can be used for
consistent iteration of the tree.
Experimental evidence shows that our algorithm outperforms a highly
tuned concurrent skip list for many access patterns, with an average of
39\% higher single-threaded throughput and 32\% higher multi-threaded
throughput over a range of contention levels and operation mixes.

\end{abstract}

\category{D.3.3}{Programming Languages}{Language Constructs and Features -- Concurrent programming structures}
\category{E.1}{Data Structures}{Trees}
\category{D.1.3}{Programming Techniques}{Concurrent Programming -- Parallel programming}

\terms
\xterms

\keywords
\xkeywords

\section{Introduction}
\label{intro}
%The proliferation of multi-core processors means that more programmers are
being thrust into the difficult world of shared memory multi-threading.
\todo{a couple sentences more of boilerplate motivation}

Software transactional memory (STM) provides a compelling alternative to
locks for managing access to shared mutable state.  STM's declarative
atomic blocks are free from deadlock, are composable, and do not
require elaborate fine-grained decomposition to yield scalability.
These benefits stem from an optimistic execution strategy that includes
rollback and retry, allowing the runtime to attempt concurrent execution
based only on a likelihood of success.  STMs bridge the gap between their
simple programming model and their speculative execution by isolating
uncommitted transactions from the rest of the system.  Most of the
difficulty in integrating an STM into a language comes from tradeoffs
between the overhead, fidelity, and complexity of this isolation.

In this paper we describe the design of CCSTM, a library-based STM for
Scala.  CCSTM deliberately sidesteps many of the performance and semantic
difficulties common in software implementations of transactional memory
by limiting its focus: CCSTM is a tool for use by parallel programmers
that wish to build algorithms and data structures using optimistic
concurrency control, not an all-encompassing concurrent programming
model or a mechanism for automatic parallelization of arbitrary code.

The most fundamental design choice for CCSTM was the decision to
implement it entirely as a Scala library.  Unlike STMs that instrument
normal loads and stores, transactional accesses in CCSTM are explicit
method calls to a Scala \xkeyword{trait}\footnote{Property accessors
may used to eliminate the syntactic overhead for many situations.}.
We refer to the resulting STM as `reference-based', because all memory
locations managed by the STM are boxed inside transactional references.
Both transactional and non-transactional access to the managed references
goes through methods implemented by instances of this \xtypeA{Ref}.

While a reference-based STM imposes a syntactic burden for simple loads and
stores, it also makes the STM more expressive.  A \xtype{Ref} provides
a first-class object that names a memory location, something that would
otherwise require reflection in Scala.  In addition, the reference provides a
convenient namespace for additional STM functionality, such as semantic
conflict detection or transformation by an associative function.
%As a
%practical benefit, the extra level of indirection provided by \xtype{Ref}
%allows multiple implementations, allowing tradeoffs between the cost of
%validation and the likelihood of rollback.

CCSTM's second departure from the typical STM interface is to bind the
transaction scope statically, rather than dynamically.  Transactional access
methods in \xtype{Ref} take an implicit parameter of type \xtype{Txn}, and
perform their accesses in the context of that transaction.  This decision
was made for pragmatic reasons, because performing a dynamically-scoped
lookup on each transactional access would be prohibitively expensive.

A useful parallel can be made between Haskell's STM [[ref]] and CCSTM.
Haskell's \xtype{TVar} corresponds to instances of type \xtype{Ref}.
\xtype{Txn} is not a monad, but it proliferates through STM-enabled
methods in exactly the same way as the \xtype{STM} monad.

In this paper, we:
\begin{packed_enum}

\item We describe CCSTM, a reference-based STM for Scala.  CCSTM focuses on
helping parallel programmers build optimistically concurrent algorithms
and data structures, while restricting itself to implementation techniques
that do not interfere with components of the system that do not use it
(Section~\ref{fig:?}).

\item We introduce \xcode{unrecordedRead}, a new STM primitive that can be coupled with
transaction life-cycle callbacks to provide semantic conflict detection.  We use
unrecorded reads to add a \xcode{map(f)} method to \xtype{Ref} that performs
conflict detection on the result of applying \xcode{f}, rather than on the
transactional value (Section~\ref{fig:?}).

\item We summarize the discussions that led from the original design goal to
the current syntax.  We point out the parts that work well and the parts that
are burdensome, and hypothesize about ways to address the latter
(Section~\ref{fig:?}).

\item We compare the performance of CCSTM to Multiverse and DeuceSTM, STMs for
the JVM that use bytecode rewriting.  We find that CCSTM's implementation as an
unprivileged library does not impose a performance penalty
(Section~\ref{fig:?}).  \todo{is this true}

\end{packed_enum}



\section{Background}
\label{background}
%The proliferation of multi-core processors means that more programmers are
being thrust into the difficult world of shared memory multi-threading.
\todo{a couple sentences more of boilerplate motivation}

Software transactional memory (STM) provides a compelling alternative to
locks for managing access to shared mutable state.  STM's declarative
atomic blocks are free from deadlock, are composable, and do not
require elaborate fine-grained decomposition to yield scalability.
These benefits stem from an optimistic execution strategy that includes
rollback and retry, allowing the runtime to attempt concurrent execution
based only on a likelihood of success.  STMs bridge the gap between their
simple programming model and their speculative execution by isolating
uncommitted transactions from the rest of the system.

STMs often attempt to mimic the syntax used for lock-based critical
regions.  The beginning and end of the atomic block are declared, and all
code that is part of the dynamic scope of the atomic region is executed
transactionally.  This has the advantage that transactional accesses
inherit the programming language's syntax for reads and writes, which
is both concise and familiar.  This familiarity is both a blessing and
a curse, however.  It creates an expectation that the STM can safely
execute arbitrary existing sequential code, which creates both semantic
and performance challenges, and it makes it more difficult for the
programmer to convey semantic information to the STM, which reduces
opportunities for compile-time checking and run-time scalability and
performance improvements.

, since it creates an expectation that the STM can
safely execute all existing sequential code.

with scoping to
guarantee pairing, 

Introducing
an STM should provide the full benefits of transactional memory for
code that uses it, without imposing a burden on existing code or syntax.
The need to isolate code executed in a transaction from the rest of the system
makes such a ``pay as you go'' STM difficult:
\begin{packed_itemize}

\item \textbf{Strong isolation:}\footnote{Strong isolation is also 
referred to as strong atomicity.} Should non-transactional memory accesses be isolated from
acceses made inside a transaction?  If so, existing code will run slower, even
if it doesn't use transactions.  If not, programmers will be exposed to a
variety of semantic pitfalls.

\item \textbf{Instrumentation overhead:} STMs 

\item \textbf{Irrevocable actions:} What happens when a transaction performs
I/O, makes a native call, or does some other action that cannot be undone?
While better alternatives may be enumerated for specific cases, a
general-purpose STM 

\item 

\end{packed_itemize}

in the decision to add STM to an existing language.


To bridge the gap between the programming model and the execution
strategy, speculative 

provides the illusion of
atomicity and isolation, while allowing 


stem from an optimistic execution model that allows speculation, rollback,
and retry.  

The STM takes responsibility for ensuring that speculative
executions are isolated from actions taken by concurrent threads.

, while isolating incorrect speculations from concurrent threads.
Isolating 

Software transactions
solve the deadlock and composability problems inherent in pessimistic
locking, while at the same time presenting a simpler interface to the
programming.


\section{Our Algorithm}
\label{algorithm}
%\input{algorithm.tex}

\section{Correctness}
\label{correctness}
%\input{correctness.tex}

\section{Performance}
\label{performance}
%\input{performance.tex}

\section{Conclusion}
\label{conclusion}
%
CCSTM demonstrates that it is possible to provide a library-only
implementation of STM for Scala.  The resulting
syntax is relatively concise and the performance is on par with a bytecode
rewriting STM.  The transactional references that encode CCSTM's interface
add some clutter to the user's code, but they also provide a natural way
for the programmer to take advantage of more sophisticated features such
as semantic conflict detection.

We found that the primary challenge in our approach was that statically
scoped transactions reduce composability.  The next step for CCSTM is
to explore a hybrid approach that uses static scopes for barriers but
dynamic scopes to coordinate nested transactions.



\appendix
\section{Code}

Source code for CCSTM is available under a BSD license from
\texttt{http://github.com/nbronson/ccstm}.

%%This is the text of the appendix, if you need one.

\acks

The authors would like to thank \todo{who} for their feedback
during the design phase of CCSTM.

This work was supported by the Stanford Pervasive Parallelism Lab,
by Dept. of the Army, AHPCRC W911NF-07-2-0027-1, and by
the National Science Foundation under grant CNS--0720905.

{\small
\bibliographystyle{abbrv}
\bibliographystyle{plainnat}
%\renewcommand{\bibfont}{\normalsize}
\bibliography{../../common/ppl}

%\begin{thebibliography}{10}
%\end{thebibliography}
}

\end{document}

