%\documentclass{sigplanconf}
\documentclass[preprint]{sigplanconf}

\usepackage{amsmath}
\usepackage{amssymb}
\usepackage{textcomp}
\usepackage[usenames]{color}
\usepackage{graphicx}
\usepackage{url}
\usepackage{rotating}
\usepackage{multirow}
\usepackage{booktabs}
\usepackage{listings}
\usepackage{array}

\newenvironment{packed_enum}{
\begin{enumerate}
  \setlength{\topsep}{0pt}
  \setlength{\itemsep}{2pt}
  \setlength{\parskip}{0pt}
  \setlength{\parsep}{0pt}
}{\end{enumerate}}

\newenvironment{packed_itemize}{
\begin{itemize}
  \setlength{\topsep}{0pt}
  \setlength{\itemsep}{2pt}
  \setlength{\parskip}{0pt}
  \setlength{\parsep}{0pt}
}{\end{itemize}}

\newcommand{\todo}[1]{{\color{red} \bf [TODO: #1 ]}}


\lstdefinelanguage{Scala}{
  morestring=[b]",
  morestring=[b]',
  morecomment=[l]{//},
  morecomment=[s]{/*}{*/},
  morekeywords={var,val,def,private,class,implicit,if,else,new,do,while,@volatile,try,catch,case,match,throw,return,extends,with}
}

\lstset{
  language=Scala,
%  basicstyle=\fontsize{7}{8}\selectfont\tt,
  basicstyle=\selectfont\tt,
  keywordstyle=\bfseries,
  numbers=left,
  numberstyle=,
  commentstyle=\itshape,
  columns=fixed,
  xleftmargin=0.25in,
  firstnumber=auto,
  escapechar=\#,
  emph={A,B,C,K,V,U,Z}, emphstyle=\bfseries\itshape,
  emph={[2]Unit,Int,Set,Map,TSet,TMap,TVar,Option,Boolean,THashSet,Txn,ConcurrentHashMap,THashMap\_Basic,THashMap\_RC,THashMap\_Weak,Pred,Token,TokRef,ConcurrentMap,CleanableRef}, emphstyle={[2]\itshape}
}



\begin{document}

\conferenceinfo{Scala Workshop,}{April 15, 2010, Lausanne, Switzerland.}
\CopyrightYear{2010}
\copyrightdata{}

\newcommand{\xtitle}[0]{CCSTM: A Library-Based STM for Scala}
\newcommand{\xterms}[0]{{Algorithms, Performance}}
\newcommand{\xkeywords}[0]{{Optimistic Concurrency, Snapshot Isolation}}
%\titlebanner{[WORK IN PROGRESS: Do Not Circulate!]}        % These are ignored unless
\preprintfooter{\xtitle}  % 'preprint' option specified.

\title{\xtitle}

\authorinfo{Nathan G. Bronson \and Kunle Olukotun}
           {Computer Systems Laboratory\\
            Stanford University}
           {\textit{\{nbronson, kunle\}@stanford.edu}}

\pdfinfo{
  /Title    (\xtitle)
  /Author   (Nathan G. Bronson and Kunle Olukotun)
  /Subject  (\xterms)
  /Keywords (\xkeywords)
}

\maketitle

\begin{abstract}

We describe the design and implementation of CCSTM, a library-only
software transactional memory (STM) implementation for Scala.
Our design philosophy is that CCSTM should be a useful tool for the
parallel programmer, rather than a parallelization mechanism for arbitrary
sequential code.  This frees us from the semantic tar pits that surround
privatization, strong atomicity and isolation, and irrevocable system calls.
It also allows us to express the STM using Scala classes and methods, a design
choice that has far-reaching consequences.

Transactional accesses in CCSTM are performed through instances that
implement the trait \texttt{Ref}.  These transactional references may be
long-lived, or may be transient accessors to bulk transactional data such
as an array.  The syntax for dereferencing \texttt{Ref} instances is a
primary pain point for the library-based approach, but the \texttt{Ref}
serves as a first-class representation of a transactionally-managed
memory location, providing a natural mechanism to express additional STM
features such as conditional waiting, non-transactional compare-and-swap,
manually-validated reads, and deferrable transformation using pure
functions.  In an additional departure from typical STM designs, the
current transactional context is passed through implicit parameters
rather than being bound to the current thread context.  This increases
the compile-time checking that can be performed, at the expense of code
clutter in method signatures.

\end{abstract}

\category{D.3.3}{Programming Languages}{Language Constructs and Features -- Concurrent programming structures}
\category{E.1}{Data Structures}{Trees}
\category{D.1.3}{Programming Techniques}{Concurrent Programming -- Parallel programming}

\terms
\xterms

\keywords
\xkeywords

\section{Introduction}
\label{intro}
%The proliferation of multi-core processors means that more programmers are
being thrust into the difficult world of shared memory multi-threading.
Software transactional memory (STM) provides a compelling alternative to
locks for managing access to shared mutable state; STM's declarative
atomic blocks are free from deadlock, are composable, and do not
require elaborate fine-grained decomposition to yield scalability.

%The benefits of memory transactions stem from an optimistic execution
%strategy that includes rollback and retry.  The ability to recover from
%error allows the runtime to
%attempt concurrent execution without a guarantee that it is possible.
%STMs bridge the gap between their simple programming model and their
%speculative execution by isolating uncommitted transactions from the
%rest of the system.  Most of the difficulty in integrating an STM into
%a language comes from tradeoffs between the overhead, fidelity, and
%complexity of this isolation.

In this paper we describe the design of CCSTM, a library-based STM
for Scala.  CCSTM deliberately sidesteps many of the 
semantic difficulties common in software implementations of transactional
memory, by limiting its focus: we view CCSTM as a domain-specific language
for use by parallel programmers that wish to build algorithms and data
structures using optimistic concurrency control.  CCSTM is not a drop-in replacement for
locks, an all-encompassing concurrent programming model, or a mechanism
for automatic parallelization of arbitrary code.

The most fundamental design choice for CCSTM was the decision to
implement it entirely as a Scala library.  Unlike STMs that transparently
instrument
all loads from and stores to shared mutable state, transactional accesses in CCSTM are explicit
method calls to a Scala \keyword{trait}\footnote{Custom accessor methods
may used to eliminate the syntactic overhead for some situations.}.
We refer to the resulting STM as `reference-based', because all memory
locations managed by the STM are boxed inside transactional references.
Both transactional and non-transactional access to the managed references
goes through methods implemented by instances of this \xtype{Ref}{A}.

While a reference-based STM imposes a syntactic burden for simple loads and
stores, it also provides advantages.  Encapsulating transactionally-managed
data eliminates the weak isolation issues that plague transparent STMs.
The \type{Ref} provides
a first-class object that names a memory location, which enables the
expression of higher-level constructs such as a simple but effective form of
semantic conflict detection.
The reference also provides a convenient
namespace for additional STM functionality.
%As a
%practical benefit, the extra level of indirection provided by \type{Ref}
%allows multiple implementations, allowing tradeoffs between the cost of
%validation and the likelihood of rollback.

CCSTM's second departure from the typical STM interface is that it binds the
transaction scope statically, rather than dynamically.  Transactional access
methods in \type{Ref} take an implicit parameter of type \type{Txn}, and
perform their accesses in the context of that transaction.
%This decision
%was made for pragmatic reasons, because performing a dynamically-scoped
%lookup on each transactional access would be prohibitively expensive.
A useful parallel can be made between Haskell's STM~\cite{harris05composable} and CCSTM.
Haskell's \type{TVar} corresponds to instances of type \type{Ref}.
\type{Txn} is not a monad, but it proliferates through STM-enabled
methods in exactly the same way as the \type{STM} monad.

In this paper:
\begin{packed_enum}

\item We describe CCSTM, a reference-based STM for Scala.  CCSTM focuses on
helping parallel programmers build optimistically concurrent algorithms
and data structures, while restricting itself to implementation techniques
that do not interfere with components of the system that do not use it
(Section~\ref{sec:ref}).

\item We introduce \code{unrecordedRead}, a new STM primitive that relaxes
read atomicity while allowing manual validation, and \code{map}, a simple
and safe way of expressing semantic conflict detection.  A transaction
that reads a refrence \code{x} via \code{x.map(f)} does not need to roll
back if \code{x} is changed concurrently but \code{f(x.get)} remains
the same (Sections~\ref{sec:unrecordedread} and~\ref{sec:map}).

\item We summarize the discussions that led from the original design goal to
the current syntax.  We point out the parts that work well and the parts that
are burdensome, and hypothesize about ways to address the latter
(Section~\ref{sec:syntax}).

\item We compare the performance of CCSTM to DeuceSTM, an STM for
the JVM that use bytecode rewriting~\cite{deucestm}.
We find that CCSTM's implementation as an
unprivileged library does not impose a significant performance penalty
(Section~\ref{sec:perf}).

\end{packed_enum}



\section{Background}
\label{background}
%The proliferation of multi-core processors means that more programmers are
being thrust into the difficult world of shared memory multi-threading.
\todo{a couple sentences more of boilerplate motivation}

Software transactional memory (STM) provides a compelling alternative to
locks for managing access to shared mutable state.  STM's declarative
atomic blocks are free from deadlock, are composable, and do not
require elaborate fine-grained decomposition to yield scalability.
These benefits stem from an optimistic execution strategy that includes
rollback and retry, allowing the runtime to attempt concurrent execution
based only on a likelihood of success.  STMs bridge the gap between their
simple programming model and their speculative execution by isolating
uncommitted transactions from the rest of the system.

STMs often attempt to mimic the syntax used for lock-based critical
regions.  The beginning and end of the atomic block are declared, and all
code that is part of the dynamic scope of the atomic region is executed
transactionally.  This has the advantage that transactional accesses
inherit the programming language's syntax for reads and writes, which
is both concise and familiar.  This familiarity is both a blessing and
a curse, however.  It creates an expectation that the STM can safely
execute arbitrary existing sequential code, which creates both semantic
and performance challenges, and it makes it more difficult for the
programmer to convey semantic information to the STM, which reduces
opportunities for compile-time checking and run-time scalability and
performance improvements.

, since it creates an expectation that the STM can
safely execute all existing sequential code.

with scoping to
guarantee pairing, 

Introducing
an STM should provide the full benefits of transactional memory for
code that uses it, without imposing a burden on existing code or syntax.
The need to isolate code executed in a transaction from the rest of the system
makes such a ``pay as you go'' STM difficult:
\begin{packed_itemize}

\item \textbf{Strong isolation:}\footnote{Strong isolation is also 
referred to as strong atomicity.} Should non-transactional memory accesses be isolated from
acceses made inside a transaction?  If so, existing code will run slower, even
if it doesn't use transactions.  If not, programmers will be exposed to a
variety of semantic pitfalls.

\item \textbf{Instrumentation overhead:} STMs 

\item \textbf{Irrevocable actions:} What happens when a transaction performs
I/O, makes a native call, or does some other action that cannot be undone?
While better alternatives may be enumerated for specific cases, a
general-purpose STM 

\item 

\end{packed_itemize}

in the decision to add STM to an existing language.


To bridge the gap between the programming model and the execution
strategy, speculative 

provides the illusion of
atomicity and isolation, while allowing 


stem from an optimistic execution model that allows speculation, rollback,
and retry.  

The STM takes responsibility for ensuring that speculative
executions are isolated from actions taken by concurrent threads.

, while isolating incorrect speculations from concurrent threads.
Isolating 

Software transactions
solve the deadlock and composability problems inherent in pessimistic
locking, while at the same time presenting a simpler interface to the
programming.


\section{Our Algorithm}
\label{algorithm}
%\input{algorithm.tex}

\section{Correctness}
\label{correctness}
%\input{correctness.tex}

\section{Performance}
\label{performance}
%\input{performance.tex}

\section{Conclusion}
\label{conclusion}
%
STM's high level programming model addresses many of the challenges of
shared memory multithreading, but to be most useful its benefits should
be provided in a pay-as-you-go fashion.  CCSTM accomplishes this by
providing transactional memory as a normal Scala library.

CCSTM uses Scala's features to embed STM as a DSL, rather than using
bytecode rewriting or VM modifications to transparently redirect loads and
stores.  Its syntax is concise, and its performance is on par
with bytecode rewriting STMs.  The references that encapsulate
transactionally-managed memory locations add some clutter to the user's
code, but they also provide a natural way for the programmer to take
advantage of more sophisticated features such as semantic conflict
detection.  The implementation is careful to avoid busy-waiting or
polling when a transaction is blocked, delivering good performance
despite contention and high multithreading levels.  CCSTM demonstrates
that a library-based STM can be usable and performant.



\appendix
\section{Code}

Source code for CCSTM is available under a BSD license from
\texttt{http://github.com/nbronson/ccstm}.

%%This is the text of the appendix, if you need one.

\acks

The authors would like to thank \todo{who} for their feedback
during the design phase of CCSTM.

This work was supported by the Stanford Pervasive Parallelism Lab,
by Dept. of the Army, AHPCRC W911NF-07-2-0027-1, and by
the National Science Foundation under grant CNS--0720905.

{\small
\bibliographystyle{abbrv}
\bibliographystyle{plainnat}
%\renewcommand{\bibfont}{\normalsize}
\bibliography{../../common/ppl}

%\begin{thebibliography}{10}
%\end{thebibliography}
}

\end{document}

