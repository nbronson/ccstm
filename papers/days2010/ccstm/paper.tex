%\documentclass{sigplanconf}
\documentclass[preprint]{sigplanconf}

\usepackage{amsmath}
\usepackage{amssymb}
\usepackage{textcomp}
\usepackage[usenames]{color}
\usepackage{graphicx}
\usepackage{url}
\usepackage{rotating}
\usepackage{multirow}
\usepackage{booktabs}
\usepackage{listings}
\usepackage{array}
\usepackage[T1]{fontenc}
\usepackage{luximono}

\newenvironment{packed_enum}{
\begin{enumerate}
  \setlength{\topsep}{0pt}
  \setlength{\itemsep}{2pt}
  \setlength{\parskip}{0pt}
  \setlength{\parsep}{0pt}
}{\end{enumerate}}

\newenvironment{packed_itemize}{
\begin{itemize}
  \setlength{\topsep}{0pt}
  \setlength{\itemsep}{2pt}
  \setlength{\parskip}{0pt}
  \setlength{\parsep}{0pt}
}{\end{itemize}}

\newcommand{\todo}[1]{{\color{red} \bf [TODO: #1 ]}}


\lstdefinelanguage{Scala}{
  morestring=[b]",
  morestring=[b]',
  morecomment=[l]{//},
  morecomment=[s]{/*}{*/},
  morekeywords={var,val,def,private,class,object,trait,implicit,if,else,new,do,while,@volatile,try,catch,case,match,throw,return,extends,with,import}
}

\lstset{
  language=Scala,
  basicstyle=\fontsize{7.5}{9}\selectfont\tt,
%  basicstyle=\fontsize{7}{8}\selectfont\tt,
%  basicstyle=\selectfont\tt,
  keywordstyle=\bfseries,
  numbers=left,
  numberstyle=,
  commentstyle=\itshape,
  columns=fixed,
  xleftmargin=0.25in,
  firstnumber=auto,
  escapechar=\#,
  emph={A,B,C,K,V,U,Z}, emphstyle=\bfseries\itshape,
  emph={[2]Unit,Int,Set,Map,TSet,TMap,TVar,Option,Boolean,Txn,Money,Account,OverdraftException,Ref,Source,Sink,Bound}, emphstyle={[2]\itshape}
}

\newcommand{\code}[1]{{\fontsize{8}{9.5}\selectfont \tt #1}}
\newcommand{\codesec}[1]{{\fontsize{10}{12}\selectfont \tt \bfseries #1}}
\newcommand{\codeft}[1]{{\fontsize{6.5}{8}\selectfont \tt #1}}
\newcommand{\type}[1]{{\code{\itshape #1}}}
\newcommand{\typesec}[1]{{\codesec{\itshape #1}}}
\newcommand{\typeparam}[1]{{\code{\bfseries #1}}}
\newcommand{\typeparamsec}[1]{{\codesec{\bfseries #1}}}
\newcommand{\xtype}[2]{{\type{#1}\typeparam{[#2]}}}
\newcommand{\xtypesec}[2]{{\typesec{#1}\typeparamsec{[#2]}}}
\newcommand{\ytype}[2]{{\type{#1}\code{[}\type{#2}\code{]}}}
\newcommand{\xxtype}[3]{{\type{#1}\typeparam{[#2,#3]}}}
%\newcommand{\keyword}[1]{{\code{\bfseries #1}}}
\newcommand{\keyword}[1]{{\code{#1}}}

\hyphenation{CC-STM}
\hyphenation{Deuce-STM}
\hyphenation{Mul-ti-verse}

\begin{document}

\conferenceinfo{Scala Workshop,}{April 15, 2010, Lausanne, Switzerland.}
\CopyrightYear{2010}
\copyrightdata{}

\newcommand{\xtitle}[0]{CCSTM: A Library-Based STM for Scala}
\newcommand{\xterms}[0]{{Algorithms, Performance}}
\newcommand{\xkeywords}[0]{{Optimistic Concurrency, Snapshot Isolation}}
%\titlebanner{[WORK IN PROGRESS: Do Not Circulate!]}        % These are ignored unless
\preprintfooter{\xtitle}  % 'preprint' option specified.

\title{\xtitle}

\authorinfo{Nathan G. Bronson \and Hassan Chafi \and Kunle Olukotun}
           {Computer Systems Laboratory\\
            Stanford University}
           {\textit{\{nbronson, hchafi, kunle\}@stanford.edu}}

\pdfinfo{
  /Title    (\xtitle)
  /Author   (Nathan G. Bronson and Hassan Chafi and Kunle Olukotun)
  /Subject  (\xterms)
  /Keywords (\xkeywords)
}

\maketitle

\begin{abstract}

We introduce CCSTM, a library-only
software transactional memory (STM) for Scala, and
give an overview of its design and implementation.
Our design philosophy is that CCSTM should be a useful tool for the
parallel programmer, rather than a parallelization mechanism for arbitrary
sequential code.  This frees us from the semantic tar pits that surround
privatization, strong isolation, and irrevocable system calls.
It also allows us to express the STM using Scala classes and methods, a design
choice that has far-reaching consequences.

Transactional accesses in CCSTM are performed through instances that
implement a trait \type{Ref}.  These transactional references may be
long-lived, or may be transient accessors to bulk transactional data such
as an array.  The syntax for dereferencing \type{Ref} instances is
a pain point for the library-based approach, but the reference-based
interface also provides benefits.  \type{Ref} serves as a first-class
representation of a transactionally-managed memory location, providing
a natural way to express additional STM features such as conditional
waiting, non-transactional compare-and-swap, manually-validated reads,
and deferrable transformation using pure functions.

In an additional departure from typical STM designs, CCSTM passes the
current transactional context through an implicit parameter, rather than
dynamically binding the transaction to the current thread.  This static
transaction scoping allows CCSTM to compete in performance with STMs that
perform bytecode rewriting, but it hinders composability.  We sketch
a potential solution to this problem that combines static scoping for
barriers and dynamic scoping for nested transactions.

\end{abstract}

\category{D.3.3}{Programming Languages}{Language Constructs and Features -- Concurrent programming structures}
\category{E.1}{Data Structures}{Trees}
\category{D.1.3}{Programming Techniques}{Concurrent Programming -- Parallel programming}

\terms
\xterms

\keywords
\xkeywords

\section{Introduction}
\label{sec:intro}
The proliferation of multi-core processors means that more programmers are
being thrust into the difficult world of shared memory multi-threading.
\todo{a couple sentences more of boilerplate motivation}

Software transactional memory (STM) provides a compelling alternative to
locks for managing access to shared mutable state.  STM's declarative
atomic blocks are free from deadlock, are composable, and do not
require elaborate fine-grained decomposition to yield scalability.
These benefits stem from an optimistic execution strategy that includes
rollback and retry, allowing the runtime to attempt concurrent execution
based only on a likelihood of success.  STMs bridge the gap between their
simple programming model and their speculative execution by isolating
uncommitted transactions from the rest of the system.  Most of the
difficulty in integrating an STM into a language comes from tradeoffs
between the overhead, fidelity, and complexity of this isolation.

In this paper we describe the design of CCSTM, a library-based STM for
Scala.  CCSTM deliberately sidesteps many of the performance and semantic
difficulties common in software implementations of transactional memory
by limiting its focus: CCSTM is a tool for use by parallel programmers
that wish to build algorithms and data structures using optimistic
concurrency control, not an all-encompassing concurrent programming
model or a mechanism for automatic parallelization of arbitrary code.

The most fundamental design choice for CCSTM was the decision to
implement it entirely as a Scala library.  Unlike STMs that instrument
normal loads and stores, transactional accesses in CCSTM are explicit
method calls to a Scala \xkeyword{trait}\footnote{Property accessors
may used to eliminate the syntactic overhead for many situations.}.
We refer to the resulting STM as `reference-based', because all memory
locations managed by the STM are boxed inside transactional references.
Both transactional and non-transactional access to the managed references
goes through methods implemented by instances of this \xtypeA{Ref}.

While a reference-based STM imposes a syntactic burden for simple loads and
stores, it also makes the STM more expressive.  A \xtype{Ref} provides
a first-class object that names a memory location, something that would
otherwise require reflection in Scala.  In addition, the reference provides a
convenient namespace for additional STM functionality, such as semantic
conflict detection or transformation by an associative function.
%As a
%practical benefit, the extra level of indirection provided by \xtype{Ref}
%allows multiple implementations, allowing tradeoffs between the cost of
%validation and the likelihood of rollback.

CCSTM's second departure from the typical STM interface is to bind the
transaction scope statically, rather than dynamically.  Transactional access
methods in \xtype{Ref} take an implicit parameter of type \xtype{Txn}, and
perform their accesses in the context of that transaction.  This decision
was made for pragmatic reasons, because performing a dynamically-scoped
lookup on each transactional access would be prohibitively expensive.

A useful parallel can be made between Haskell's STM [[ref]] and CCSTM.
Haskell's \xtype{TVar} corresponds to instances of type \xtype{Ref}.
\xtype{Txn} is not a monad, but it proliferates through STM-enabled
methods in exactly the same way as the \xtype{STM} monad.

In this paper, we:
\begin{packed_enum}

\item We describe CCSTM, a reference-based STM for Scala.  CCSTM focuses on
helping parallel programmers build optimistically concurrent algorithms
and data structures, while restricting itself to implementation techniques
that do not interfere with components of the system that do not use it
(Section~\ref{fig:?}).

\item We introduce \xcode{unrecordedRead}, a new STM primitive that can be coupled with
transaction life-cycle callbacks to provide semantic conflict detection.  We use
unrecorded reads to add a \xcode{map(f)} method to \xtype{Ref} that performs
conflict detection on the result of applying \xcode{f}, rather than on the
transactional value (Section~\ref{fig:?}).

\item We summarize the discussions that led from the original design goal to
the current syntax.  We point out the parts that work well and the parts that
are burdensome, and hypothesize about ways to address the latter
(Section~\ref{fig:?}).

\item We compare the performance of CCSTM to Multiverse and DeuceSTM, STMs for
the JVM that use bytecode rewriting.  We find that CCSTM's implementation as an
unprivileged library does not impose a performance penalty
(Section~\ref{fig:?}).  \todo{is this true}

\end{packed_enum}



\section{Background}
\label{sec:library}

An experimental feature such as software transactional memory should
strive to impose only negligible costs on code that does not use it.
Runtime performance costs are the most obvious, but extra complexity in
the compiler, libraries, and language rules should be minimized.
A pay-as-you-go philosophy facilitates incremental adoption, it allows
multiple implementations to coexist, and it reduces the penalty for
failure.

One popular and reasonable interface design for transactional memory
is to mimic lock-based critical regions.  Users of such an STM declare
the beginning and the end of an atomic block, and all memory accesses
that occur within the dynamic scope of the block are transparently
redirected to the STM.  For a VM language like Scala this redirection
can be introduced by the VM's JIT, by bytecode rewriting at class
load time, or during the initial
compilation of the high-level language.  The dynamic scoping of such an approach,
however, means that it is generally not possible to
limit instrumentation to only classes that are used in an atomic block.
An STM that is deeply integrated into the VM's JIT can minimize the
performance and code bloat impacts of the instrumentation by performing it
%todo, add citation
lazily, but the required engineering effort to add this support
to a production quality VM is prohibitively large.  Instrumentation of
the bytecode at compilation or class loading has the lowest engineering
cost, but results in two copies of each method.
This is the strategy adopted by the Multiverse~\cite{multiverse} and
Deuce STM~\cite{deucestm} STMs for the Java language.
While this cost may eventually be considered acceptable, it places a
high hurdle to integration into Scala's standard library.
An additional drawback of
an instrumentation approach is that it is not composable.  If module
\textit{A} is constructed with the STM \textit{S} and module \textit{B}
is constructed with the STM \textit{T}, then \textit{A} and \textit{B}
can't be used in the same program.

The alternative approach adopted by CCSTM is to require the programmer to perform
explicit calls to the STM.  While less convenient for simple uses, this
limits performance side-effects on code that does not use atomic blocks, and it
allows the STM to be constructed entirely as an unprivileged library.
When coupled with an STM design that does not assume it is managing all
threads, the result is a pay-as-you-go transactional memory suitable
for experimentation and incremental adoption.

Scala's flexible syntax makes a library-only STM tractable.  Operator
overloading makes transactional loads and stores concise, and implicit
parameters allow the current transaction context to be statically
threaded through the code without explicitly including it in each call.
The resulting STM can be considered an embedded DSL
for optimistic concurrency.


%\section{\xtypesec{Ref}{A} and \xtypesec{Ref.Bound}{A}}
\section{CCSTM's interface}
\label{sec:ref}
As a recurring example of the ways in which CCSTM allows optimistic concurrency
to be expressed, consider a class that encapsulates the balance of a
checking account\footnote{This example is from the bank benchmark included
in DeuceSTM~\cite{deucestm}.}.  Absent any concurrency control, we might
write the code in Figure~\ref{fig:example:nosync}.  (Here \type{Money}
is an immutable numeric type suitable for representing quantities of
a currency.)  Adding pessimistic concurrency control to this code by
locking accesses to \type{Account} instances is not straightforward,
because both the source and destination account must be locked during
a \code{transfer}.  Unless a global lock order is followed this can
easily lead to deadlock.

\begin{figure}
\begin{lstlisting}{name=Code}
class Account(initialBalance: Money) {
  private var _balance = initialBalance

  def balance: Money = _balance

  def deposit(amount: Money): Unit = {
    assert(amount >= 0)
    _balance += amount
  }

  def withdraw(amount: Money): Unit = {
    assert(amount >= 0)
    if (_balance < amount)
      throw new OverdraftException
    _balance -= amount
  }
}

object Account {
  def transfer(src: Account, dst: Account,
        amount: Money): Unit = {
    src.withdraw(amount)
    dst.deposit(amount)
  }
}
\end{lstlisting}

\caption{Code that performs an account transfer without
any locking or other concurrency control.}

\label{fig:example:nosync}
\end{figure}

\begin{figure}
\begin{lstlisting}{name=Code}
class Account(initialBalance: Money) {
  private val _balance = Ref(initialBalance)

  def balance: Source[Money] = _balance

  def deposit(amount: Money
        )(implicit txn: Txn): Unit = {
    assert(amount >= 0)
    _balance := !_balance + amount
  }

  def withdraw(amount: Money
        )(implicit txn: Txn): Unit = {
    assert(amount >= 0)
    if (!_balance < amount)
      throw new OverdraftException
    _balance := !_balance - amount
  }
}

object Account {
  def transfer(src: Account, dst: Account,
        amount: Money): Unit = {
    new Atomic { def body {
      src.withdraw(amount)
      dst.deposit(amount)
    }}.run()
  }
}
\end{lstlisting}

\caption{An atomic balance transfer function implemented with CCSTM.}

\label{fig:example:ccstm}
\end{figure}

The most fundamental data type in CCSTM is \xtype{Ref}{A}, which mediates
access to an STM-managed value.  Read-only methods are separated into
a covariant \type{Source} trait and write-only methods are separated
into a contravariant \type{Sink} trait.  Instances of \type{Ref} are
independent of the current transactional context, so the context is
passed during each method call via an implicit parameter.

\begin{figure*}
  \centering
  \includegraphics[clip=true,width=5.5in]{build/refs_class_uml}

\caption{Traits that provide access to an STM-managed memory
location.  Transactional access can occur through either \type{Ref}
or a \type{Ref.Bound} returned from \type{Ref}\code{.bind},
non-transactional access occurs through a \type{Ref.Bound} returned
from \type{Ref}\code{.nonTxn}.  \xtype{Source}{+A} and \xtype{Sink}{-A}
decompose the covariant and contravariant operations of \xtype{Ref}{A}.}

\label{fig:refsclasses}
\end{figure*}

Non-transactional access to the contents of a reference are provided by
a view returned by \code{nonTxn}.  This view implements methods that
parallel those of the reference, but that don't require a \type{Txn}.
We say that the view is \textit{bound} to the non-transactional
context, so the view trait is named \type{Ref.Bound}.  Views may
also be bound to a transactional context via \type{Ref}\code{.bind}.
Figure~\ref{fig:refclasses} shows the \keyword{extends} relationship
between the traits that implement unbound and bound references, and 
some of their methods.

Bound views for non-transactional access create a syntactic difference
between transactional and non-transactional reads and writes.
This allows the expert programmer to selectively relax isolation by
performing a non-transactional access inside an atomic block, without
requiring an escape action~\cite{escapeaction}.  The non-isolated access
is visually differentiated by including the token \code{nonTxn}.  In
Section~\ref{sec:unrecordedread} we will introduce \code{unrecordedRead},
a way of relaxing isolation for reads while retaining the ability to
manually validate.


Ref eq vs ==

operations


\section{CCSTM's implementation}
\label{sec:impl}

CCSTM's implementation is modeled on SwissTM~\cite{swisstm}.  Version
management is lazy, but write permission is acquired eagerly.  Time-stamps
are allocated 51 bits, making CCSTM immune from counter overflow in all
but the most demanding production environments.

\subsection{Meta-data indirection}

Meta-data for a managed memory location consists of a single \keyword{long}.
It is assumed that each memory location maps to a unique meta-data value, but
not vice versa.  This allows objects with multiple fields to use
a single piece of meta-data, and it allows arrays to choose a variety of
granularities of conflict detection.  While some optimizations are possible for
situations where the data-to-meta-data mapping is one-to-one, in informal
experiments we found that the benefits were smaller than the additional
indirection costs.

\type{Ref}s perform their accesses to both data and meta-data through
methods of an internal trait called a \type{Holder}.  This indirection
allows multiple storage strategies to be easily provided, which can
yield an important reduction in the number of live objects in the VM.
For example, if the static or manifest type of the initial value is known
to be an \keyword{int}, then the \type{Ref} factory method will return
a reference whose holder stores the value in an unboxed form.  As a
more extreme example, CCSTM provides a transactional array-like class
that internally uses one array for values and one array for meta-data,
eliminating the $n$ intermediate objects that would be required by an
\ytype{Array}{\xtype{Ref}{A}}.  These storage optimizations help CCSTM
keep the implementation costs of its library-based approach near those
of an instrumenting STM.

\subsection{Global time-stamp optimizations}

To reduce contention on the shared time-stamp, CCSTM uses TL2's GV6
scheme~\cite{dice06tl2}.  This mechanism is based on the observation
that, while committed values must be given a time-stamp later than the
version clock that was present at the beginning of the commit, it is
not required that the global clock is actually advanced.  Advancing the
global clock reduces the need for validation in later transactions,
but when many threads are using the STM, this goal is satisfied even if
only a fraction of transactions attempt to advance the current time.

CCSTM performs a novel additional optimization to reduce the overhead
of non-transactional accesses.  Unlike a transaction, a solitary
strongly-isolated read or write in a TL2-style STM does not need to
sample the global clock to provide opacity.  This means that we can allow
a sequence of non-transactional writes to advance a reference's time-stamp to
an arbitrary point in the future, without advancing the global time-stamp.
If a transaction attempts to read such a far-future value it handles
it via the normal GV6 mechanism, by advancing the global time-stamp and
then revalidating.  To limit the potential impact of these booby-trapped
references, we only allow non-transactional writes to advance time-stamps
a limited distance into the future.  Even a small window (CCSTM defaults
to 8) dramatically reduces contention on the global time-stamp.

\subsection{Avoiding starvation}

Optimistic concurrency control is vulnerable to the \textit{starving
elder} problem, in which a large transaction can never be committed because
it is continually violated by small transactions.  CCSTM uses a simple
contention manage scheme to prevent this.  Each execution attempt is
assigned a random priority that is used to resolve write-write conflicts.
If a transaction has not yet begun to commit, then a higher priority
transaction may doom it and steal its locks.  In addition, transactions
that have already failed several times enter a `barging' mode in which
they acquire write permission during reads.  The result is that even
large transactions will eventually succeed, because they will eventually
receive the highest priority in the system.

\subsection{Polite blocking}

An important design goal for CCSTM is support for incremental use inside
a larger application.  This means that exponential back-off is not a
suitable mechanism for blocking.  Many STMs target parallel speedups
for CPU-bound applications.  Assumed (often implicitly) is that all
threads belong to the STM and that the number of software threads can be
chosen to match the number of hardware execution contexts.  CCSTM makes
neither of these assumptions, and so takes care to block using the normal
synchronization primitives of the underlying VM.

Blocking may be required to obtain write permission, or because of an
explicit use of the \code{retry} primitive.  Writers and waiters must
agree on a condition variable that will be used to signal that the
waiter should re-attempt whatever action led to their choice to block.
If the set of condition variables is too small, there will be many
spurious wakeups.  If the set is too large, then transaction commit may need to
perform a large amount of extra work.

Accesses that are blocked by another transaction await notification
on the \type{Txn} instance itself.  No such instance is available for
threads blocked by a non-transactional write, or that are performing a
conditional retry, so the system also maintains 64 lists we refer to as
`wakeup channels'.  These channels contain a list of pending wakeups,
which are single-shot gates (compare to Java's \type{CountDownLatch}
with a count of 1).  Each memory location is associated with a wakeup
channel by hashing its identity.  To await the modification of a
memory location, a thread enqueues a new pending wakeup instance,
sets a `wakeup pending' bit in the location's meta-data, rechecks the
blocking condition, and then puts itself to sleep on the gate.  If an
update notices the wakeup pending bit, it triggers and removes all of
the pending wakeups for the corresponding channel.  A thread may wait on
multiple memory locations simultaneously by enqueuing its pending wakeup
instance to multiple channels.  The choice of 64 wakeup channels makes
it easy to accumulate the effects of a transaction in a \keyword{long}.
If a system makes extremely heavy use of the \code{retry} mechanism by
having many blocked threads, a larger number of channels may be required.

\subsection{JVM versus CLR}

Scala is designed to target both the JVM and the CLR virtual
machines.  In its current implementation, CCSTM uses the following
classes from \type{java.util.concurrent} to implement atomic
compare-and-swap for fields and array elements: \type{AtomicInteger},
\type{AtomicLong}, \type{AtomicLongArray}, \type{AtomicReferenceArray},
\type{At\-om\-ic\-LongFieldUpdater}, and \type{AtomicReferenceField\-Up\-dat\-er}.
The authors are not experts in the CLR memory model, but we believe
that it would be straightforward to retarget CCSTM to the CLR by using
methods in \type{System::Threading::Interlocked}.



\section{Discussion}
\label{sec:discussion}

STM research is mature enough that the most difficult design decisions
for CCSTM were all found in the interface.

\subsection{Read barrier syntax}
\label{sec:syntax}

The most difficult syntactic choice was the method name for a
transactional read.  Three concise alternatives were considered:

\begin{packed_enum}

\item An implicit conversion from \xtype{Ref}{A} to \typeparam{A} -- This is
the most concise, but it interferes with further implicit conversions and type
inference for generic methods or data structures.  If \code{x} is a
\ytype{Ref}{String} holding \code{"foo"}, for example, \code{("foo" == x)}
would return false.

\item \code{unary\_!} -- A unary operator is the most concise explicit
way of denoting a read, and prefix forms of these are the only ones
that don't trigger Scala's line merging heuristic.  Initially we
settled on a \code{!} prefix for reads.  This works well for arithmetic
expressions, but it can be confusing when used in a conditional test,
as in \code{(!x == "foo")}.  It also does not chain well, requiring
extra parentheses: \code{(!x).length}.  Because of these problems, we
found ourself often reverting to the more verbose forms \code{(x.get ==
"foo")} and \code{x.get.length}.

\item \code{apply()} -- This is the only operator-like suffix method
that can safely occur at the end of the line, which means that it chains
properly in complex operations.  We initially avoided using it for read
barriers, because \code{()} is often used in Scala to help draw attention
to side-effecting methods.  Although read barriers do not themselves
perform a mutation, their access to shared mutable state also requires
extra care.  The current CCSTM code base uses \code{apply()} for concise
representation of read barriers.

\end{packed_enum}

For fields that are almost always accessed inside an atomic block, we
have also found it convenient to create transactional accessor methods.
The full \type{Ref} is published via a longer name for non-transactional
or advance operations, while the basic property name is available within
a transaction using Scala's basic field syntax:
\lstset{numbers=none}
\lstset{xleftmargin=0.125in}
\begin{lstlisting}
class Node {
  val nextRef: Ref[Node] = ..
  def next(implicit txn: Txn): Node = nextRef()
  def next_=(v: Node)(implicit txn: Txn) {
    nextRef := v 
  }
}
\end{lstlisting}
\lstset{xleftmargin=0.25in}
\lstset{numbers=left}

\color{green}
\subsection{Nested transactions}

CCSTM currently does not support nested atomic blocks, an important
omission.  A larger problem, though, is that the static scoping
of transactions would make composing such blocks difficult.  If a
method \code{m} needs a transaction internally, should it add an
\keyword{implicit} \type{Txn} parameter?  If it doesn't, then it
cannot be composed.  If it does, then the caller must always provide
an atomic block.  If the programmer wishes to provide both a simple
and a composable version of the method then he must come up with
two names.

The need for two versions of \code{m} parallels the transactional and
non-transactional access to a \type{Ref}, which was resolved there by the
\type{Ref} $\leftrightarrow$ \type{Ref.Bound} correspondence.  Rather than
coming up with two method names, the same method name can be used in
two classes.  \type{Ref}\code{.nonTxn} and \type{Ref.Bound}\code{.unbind}
convert from an instance suitable for one context to the other.  While it
may be tolerable to manage these parallel classes in the library itself,
it seems burdensome to ask the programmer to perform a similar task.

A possible solution is to provide dynamic scoping for the declaration
of atomic blocks, but static scoping for barriers.  Because nested
transactions are likely to be less common than barriers, there is less
performance motivation for avoiding the thread-local lookup.

It seems dangerous to mix static and dynamic scoping, but we notice that
for all but the most subtle uses, the static and dynamic scopes should be
identical.  This means that we can provide an execution mode that checks
the
static scopes against the dynamic ones, and triggers an exception 
if they don't match.  This checking mode would be slower on a per-barrier
basis, so like asserts it might be disabled during production use.

For applications that don't spend a large portion of their time in barriers, we
should also consider providing full dynamic scoping for transactions.
Interestingly, this could be enabled in a per-module fashion by providing an
implicit function that performs the required thread-local lookup.  The result
might look like:
\lstset{numbers=none}
\begin{lstlisting}
class Account ... {
  import STM.dynamic

  def deposit(amount: Money) {
    assert(amount >= 0)
    _balance := _balance() + amount
  }
}
\end{lstlisting}
\lstset{numbers=left}
If \code{deposit} were executed outside a transaction a runtime error would be
generated, rather than the compile-time error produced by the existing CCSTM.

Future: Partial rollback

Future: Collections

Future: Plugin

Future: @specialized

\color{black}

*Simple got more complicated
*Complicated got easier
*Implicit txn


Green Marker Notes:
~~~~~~~~~~~~~~~~~~~
txnal reference as a trait
    w/ multiple implementations
  - Clunky base syntax
  - Nice extended syntax
  - Optimized back ends (StripedIntRef, TIntRef,...)

Ref interface to collections
  - key is handle
  - TArray
  - THashMap

TxnFieldUpdater w/plugin

Syntax for atomic, orElse
  -tranformz
  -goal: good nonTxn support
        composition of ops on txn objects, 
        not actions in a transaction on all objects

%% \section{Performance}
%% \label{performance}
%% %\input{performance.tex}

%% \section{Conclusion}
%% \label{conclusion}
%% %
CCSTM demonstrates that it is possible to provide a library-only
implementation of STM for Scala.  The resulting
syntax is relatively concise and the performance is on par with a bytecode
rewriting STM.  The transactional references that encode CCSTM's interface
add some clutter to the user's code, but they also provide a natural way
for the programmer to take advantage of more sophisticated features such
as semantic conflict detection.

We found that the primary challenge in our approach was that statically
scoped transactions reduce composability.  The next step for CCSTM is
to explore a hybrid approach that uses static scopes for barriers but
dynamic scopes to coordinate nested transactions.



\appendix
\section{Code}

Source code for CCSTM is available under a BSD license from
\textsf{http://github.com/nbronson/ccstm}.

%%This is the text of the appendix, if you need one.

\acks

The authors would like to specifically thank Daniel Spiewak and Peter
Veentjer for their helpful feedback during the design phase of CCSTM.

This work was supported by the Stanford Pervasive Parallelism Lab,
by Dept. of the Army, AHPCRC W911NF-07-2-0027-1, and by
the National Science Foundation under grant CNS--0720905.

{
%\small
\bibliographystyle{abbrv}
%\bibliographystyle{plainnat}
%\renewcommand{\bibfont}{\normalsize}
\bibliography{../../common/ppl}

%\begin{thebibliography}{10}
%\end{thebibliography}
}

\end{document}

